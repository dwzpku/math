\documentclass[leqno,12pt]{article}
\usepackage[a4paper,margin=0.5in,footskip=0.5in,bottom=0.75in]{geometry}
\usepackage{authblk}
\usepackage{amsrefs}
\usepackage{graphicx,color}

\makeatletter
\DeclareSizeFunction{sub}{\sub@sfcnt\@font@info}
\makeatother
\usepackage{stix2}
\usepackage{amsmath}
\usepackage{mathtools}
\usepackage{bm}
\usepackage{braket}

\newcommand{\Emph}[1]{\noindent\textsf{\Large #1\ \ }}
\newcommand{\Int}[1]{\left[#1\right]}

\usepackage{amsthm}
\renewcommand{\qedsymbol}{\tiny$\blacksquare$}
\newcommand{\noqed}{\renewcommand{\qed}{}}
\newcommand*{\newqedtheoremstar}[1]{%
  \AddToHook{env/#1/begin}{\pushQED{\qed}}%
  \AddToHook{env/#1/end}{\popQED}%
  \newtheorem*{#1}}
\newcommand*{\newqedtheorem}[1]{%
  \AddToHook{env/#1/begin}{\pushQED{\qed}}%
  \AddToHook{env/#1/end}{\popQED}%
  \newtheorem{#1}}
\newtheoremstyle{plain}% name
  {6pt}%        Space above
  {6pt}%        Space below
  {\normalfont}% Body font
  {}%           Indent amount
  {\bfseries}%  Head font
  {.}%           Punctuation after head
  {1.0em}%        Space after head
  {\thmname{#1}\thmnumber{ #2}\thmnote{. \ \itshape #3}}% Head spec
\newqedtheoremstar{definition}{Definition}
\newqedtheorem{lemma}{Lemma}[section]
\newqedtheorem{theorem}{Theorem}[section]
\newqedtheorem{corollary}{Corollary}[theorem]
\newenvironment{solution}{\begin{proof}[解]}{\end{proof}}

\usepackage{titlesec}
\titleformat{\section}{\sffamily\Huge}{\thesection}{0.5em}{}
\titleformat{\subsection}{\sffamily\bfseries\Large}{\thesubsection}{0.5em}{}
\titleformat{\subsubsection}{\sffamily\bfseries\large}{}{0pt}{}
\usepackage[scheme=plain,UTF8,fontset=windows]{ctex}

\usepackage{hyperref}
\begin{document}
\title{初等数论笔记}
\author{邓卫真}
\affil{北京大学物理学院}
\date{2025年11月6日开始}
\maketitle

\section{The Integers}

\subsection{Numbers and Sequences}

\subsubsection{Numbers}

\Emph{The Well-Ordering Property} Every nonempty set of positive
integers has a least element.

\begin{definition}
  The real number \(r\) is \emph{rational} if there are integers \(p\) and
  \(q\), with \(q\ne0\), such that \(r=p/q\). If \(r\) is not rational, it is
  said to be \emph{irrational}.
\end{definition}

\begin{theorem}
  \(\sqrt2\) is irrational.
\end{theorem}

\begin{definition}
  A number \(\alpha\) is \emph{algebraic} if it is the root of a polynomial with
  integer coefficients; that is, \(\alpha\) is algebraic if there exist
  integers \(a_0, a_1, \ldots, a_n\) such that
  \(a_n\alpha^n+a_{n-1}\alpha^{n-1}+\cdots+a_0=0.\) The number
  \(\alpha\) is called \emph{transcendental} if it is not algebraic.
\end{definition}

\subsubsection{The Greatest Integer Function}

\begin{definition}
  The \emph{greatest integer} in a real number \(x\), denoted by
  \([x]\), is the largest integer less than or equal to \(x\). That is,
  \([x]\) is the integer satisfying
  \[
    [x]\le x <[x]+1. \qedhere
  \]
\end{definition}

\begin{definition}
  The \emph{fractional part} of a real number \(x\), denoted by
  \(\{x\}\), is the difference between \(x\) and the greatest integer in
  \(x\), namely, \([x]\). That is, \(\{x\}=x-[x]\).
\end{definition}

\subsubsection{Diophantine Approximation}

\begin{theorem}[The Pigeonhole Principle]
  If \(k+1\) or more objects are placed into \(k\) boxes, then at least
  one box contains two or more of the objects.
\end{theorem}

\begin{theorem}[The Dirichlet's Approximation Theorem]
  If \(\alpha\) is a real number and \(n\) is a positive integer, then there
  exist integers \(a\) and \(b\) with \(1\le a \le n\) such that
  \(|a\alpha-b|<1/n\).
\end{theorem}

\subsubsection{Sequences}

\begin{definition}
  A \emph{geometric progression} is a sequence of the form \(a\),
  \(ar\), \(ar^2\), ..., \(ar^k\), ..., where \(a\), the \emph{initial term}, and
  \(r\), the \emph{common ratio}, are real numbers.
\end{definition}

\begin{definition}
  A set is \emph{countable} if it is finite or it is infinite and there
  exists a one-to-one correspondence between the set of positive
  integers and the set. A set that is not countable is called
  \emph{uncountable}.
\end{definition}

\begin{theorem}
  The set of rational numbers is countable.
\end{theorem}

\subsection*{Exercises}

\begin{itemize}
\item [5.] Use the well-ordering property to show that \(\sqrt3\) is
  irrational.
  \begin{solution}
    假设 \(\sqrt{N}=p/q\), 则 \(q\sqrt{N}=p\) 为整
    数. 由 well-ordering 性质, 设 \(s\) 为整数集合
    \[
      Q=\Set{q\in \mathbb{Z}^+|q\sqrt{N}\in \mathbb{Z}^+}
    \]
    内最小正整数. 考虑
    \(s'=\Set{\sqrt{N}}s=\sqrt{N}s-\Int{\sqrt{N}}s\in
    \mathbb{Z}\) 非负. 而\(s'\sqrt{N}=sN-\Int{\sqrt{N}}s\sqrt{N}\) 也为非负整
    数. 因为 \(\Set{\sqrt{N}}<1\), 所以 \(0\le s'<s\). 因 \(s\) 在 \(Q\) 中最小, 所
    以 \(s'=0\). 即 \(\sqrt{N}\) 必须为整数.
  \end{solution}
\item[20.] Show that if \(m\) is a positive integer, then
  \[
    [mx]=[x[x+(1/m)]+[x+(2/m)]+\cdots+[x+(m-1)/m]
  \]
  whenever \(x\) is a real number.
  \begin{solution}
    令 \(mx=n+\alpha\), \(n=[mx]\), \(\alpha=\{mx\}\). 再令 \(n=mp+q\),
    \(p=[n/m]\), \(0\le q<m\) 为非负整数. 考虑
    \[
      \sum_{k=0}^{m-1}[\alpha/m+k/m]=0.
    \]
    即
    \[
      \sum_{k=0}^{m-1}[x-n/m+k/m]=\sum_{k=0}^{m-1}[x-p-q/m+k/m]=0.
    \]
    计算
    \begin{align*}
      \sum_{k=0}^{q-1}[x-p-q/m+k/m]=&\sum_{k=0}^{n-1}\left\{ [x+1-q/m+k/m]-1 \right\}\\
      =&\sum_{k=m-q}^{m-1}[x+k/m] - q(p+1), \\
      \sum_{k=q}^{m-1}[x-p-q/m+k/m]=&\sum_{k=0}^{m-q-1}[x+k/m] - p(m-q).
    \end{align*}
    得
    \[
      \sum_{k=0}^{m-1}[x+k/m] =mp+q= n = [mx]. \qedhere
    \]
  \end{solution}
\item Show that every positive integer occurs exactly once in the
  spectrum sequence of \(\alpha\) or in the spectrum of \(\beta\) if and only if
  \(\alpha\) and \(\beta\) are positive irrational numbers such that
  \(1/\alpha+1/\beta=1\).
  \begin{solution}
    假设 \(1/\alpha+1/\beta=1\). 首先, 要使序列 \(m\alpha\) 和 \(m\beta\) 不相交,
    \(\alpha\) 和 \(\beta\) 都必须是无理数. 然后, 对正整数 \(k\), 定义 \(N(k)\) 为序
    列 \(m\alpha\) 和 \(m\beta\) 中小于 \(k\) 的项
    数. 则 \(N(k)=[k/\alpha]+[k/\beta]\). 因 \(k/\alpha\) 和 \(k/\beta\) 均非整
    数, 得 \(k/\alpha-1<[k/\alpha]<k\alpha\) 及
    \(k/\beta-1<[k/\beta]<k/\beta\). 相加得 \(k-2<N(k)<k\). 这样 \(N(k)=k-1\).
  \end{solution}
\end{itemize}

\subsection{Sums and Products}

The following notation represents the sum of the numbers
\(a_1,a_2, \ldots, a_n\):
\[
  \sum_{k=1}^na_k=a_1+a_2+ \cdots+a_n.
\]

The products of the numbers \(a_1,a_2, \ldots, a_n\) is denoted by
\[
  \prod_{j=1}^n a_j=a_1a_2 \cdots a_n.
\]

\subsection{Mathematical Induction}

\begin{theorem}[The Principle of Mathematical Induction]\noqed
  A set of positive integers that contains the integer \(1\), and that
  has the property that, if it contains the integer \(k\), then it also
  contains \(k+1\), must be the set of all positive integers.
\end{theorem}
\begin{proof}
  The principle of mathematical induction follows from the
  well-ordering principle.
\end{proof}

\begin{theorem}[The Second Principle of Mathematical Induction]
  A set of positive integers that contains the integer \(1\), and that
  has the property that, for every positive integer \(n\), if it
  contains all the positive integers \(1,2,\ldots,n\), then it also contains
  the integer \(n+1\), must be the set of all positive integers.
\end{theorem}

\subsubsection{Recursive Definition}

\begin{definition}
  We say that the function \(f\) is \emph{defined recursively} if the value
  of \(f\) at \(1\) is specified and if for each positive integer
  \(n\) a rule is provided for determining \(f(n+1)\) from \(f(n)\).
\end{definition}

The principle of mathematical induction can be used to show that a
function that is defined uniquely at each positive integer.

\subsection*{Exercises}

\begin{itemize}
\item [35.] A \emph{unit fraction} is a fraction of the form \(1/n\), where
  \(n\) is a positive integer. Because the ancient Egyptian represented
  fractions as sum of distinct unit fractions, such sums are called
  \emph{Egyptian fractions}. Show that every rational number \(p/q\), where
  \(p\) and \(q\) are integers with \(0<p<q\), can be written as a sum of
  distinct unit fractions, that is, as an Egyptian fraction.
  \begin{proof}
    命题在 \(p=1\) 时显然真. 我们对 \(p\) 作强递推. 设命题对 \(p<n\) 的所有分
    数 \(0<p/q<1\)都成立. \(p=n>1\) 时, 设 \(1/s\) 为小于 \(p/q\) 的最大单位分
    数, 则 \(1/(s-1)>p/q>1/s\). 相减得剩余分数 \(p/q-1/s=(ps-q)/qs\). 由不
    等式得 \(ps-q>0\) 和 \(p(s-1)-q<0\). 因此剩余分数的分子 \(ps-q<p=n\).
  \end{proof}
\end{itemize}

\subsection{The Fibonacci Numbers}

\begin{definition}
  The \emph{Fibonacci sequence} is defined recursively by \(f_1=1\),
  \(f_2=1\), and \(f_n=f_{n-1}+f_{n-2}\) for \(n\ge 3\). The terms of this
  sequence are called the \emph{Fibonacci numbers}.
\end{definition}

\subsubsection{How Fast Do the Fibonacci Numbers Grow?}

\begin{theorem}
  Let \(n\) be a positive integer and let
  \(\alpha=\frac{1+\sqrt5}{2}\) and
  \(\beta=\frac{1-\sqrt5}{2}\). Then the \(n\)th Fibonacci number
  \(f_n\) is given by
  \[
    f_n=\frac1{\sqrt5}(\alpha^n-\beta^n). \qedhere
  \]
\end{theorem}

\subsubsection*{Computations and Explorations}

\begin{itemize}
\item [7.] A surprising theorem states that the Fibonacci numbers are
  the positive values of the polynomial
  \(2xy^4+x^2y^3-2x^3y^2-y^5-x^4y+2y\) as \(x\) and \(y\) range over all
  nonnegative integers.\cite{Jones75}
\end{itemize}

\subsection{Divisibility}

\begin{definition}
  If \(a\) and \(b\) are integers with \(a\ne0\), we say that \(a\) \emph{devides}
  \(b\) if there is an integer \(c\) such that \(b=ac\). If \(a\) divides
  \(b\), we also say that \(a\) is a \emph{divisor} or \emph{factor} of
  \(b\) and that \(b\) is a \emph{multiple} of \(a\).
\end{definition}

\begin{theorem}
  If \(a,b\) and \(c\) are integers with \(a\mid b\) and \(b\mid c\), then \(a\mid c\).
\end{theorem}

\begin{theorem}
  If \(a,b,m\), and \(n\) are integers, and if \(c\mid a\) and
  \(c\mid b\), then \(c\mid(ma+nb)\).
\end{theorem}

\begin{theorem}[The Division Algorithm]
  If \(a\) and \(b\) are integers such that \(b>0\), then there are unique
  integers \(q\) and \(r\) such that \(a=bq+r\) with \(0\le r<b\).
\end{theorem}

\begin{definition}
  If the remainder when \(n\) is divided by \(2\) is \(0\), then
  \(n=2k\) for some integer \(k\), and we say that \(n\) is \emph{even}, whereas
  if the remainder when \(n\) is divided by \(2\) is \(1\), then
  \(n=2k+1\) for some integer \(k\), and we say that \(n\) is \emph{odd}.
\end{definition}

\subsubsection{Greatest Common Divisors}

\begin{definition}
  The \emph{greatest common divisor} of two integers \(a\) and \(b\), which
  are not both \(0\), is the largest integer that divides both \(a\) and
  \(b\).
\end{definition}

\begin{definition}
  The integers \(a\) and \(b\), with \(a\ne0\) and \(b\ne0\), are \emph{relatively
    prime} if \(a\) and \(b\) have greatest common divisor \((a,b)=1\).
\end{definition}

\section{Integer Representations and Operations}

\subsection{Representations of Integers}

\begin{theorem}
  Let \(b\) be a positive integers with \(b>1\). Then every positive
  integer \(n\) can be written uniquely in the form
  \[
    n=a_kb^k+a_{k-1}b^{k-1}+\cdots+a_1b+a_0,
  \]
  where \(k\) is a nonnegative integer, \(a_j\) is an integer with
  \(0\le a_j\le b-1\) for \(j=0,1,\ldots,k\), and the initial coefficient
  \(a_k\ne0\).
\end{theorem}

\begin{corollary}
  Every positive integer may be represented as the sum of distinct
  powers of \(2\).
\end{corollary}

\subsection{Computer Operations with Integers}

\begin{definition}
  An \emph{algorithm} is a finite set of precise instructions for
  performing a computation or for solving a problem.
\end{definition}

\subsection{Complexity of Integer Operations}

\begin{definition}
  If \(f\) and \(g\) are functions taking positive values, defined for all
  \(x\in S\), where \(S\) is a specified set of real numbers, then
  \(f\) is \(O(g)\) if there is a positive constant \(K\) such that
  \(f(x)<Kg(x)\) for all sufficiently large \(x\in S\). (Normally, we take
  \(S\) to be the set of positive integers, and we drop all reference to
  \(S\).)
\end{definition}

\begin{theorem}
  If \(f\) is \(O(g)\) and \(c\) is a positive constant, then \(cf\) is \(O(g)\).
\end{theorem}

\begin{theorem}
  If \(f_1\) is \(O(g_1)\) and \(f_2\) is \(O(g_2)\), then \(f_1+f_2\) is
  \(O(g_1+g_2)\), and \(f_1f_2\) is \(O(g_1g_2)\).
\end{theorem}

\begin{corollary}
  If \(f_1\) and \(f_2\) are \(O(g)\), then \(f_1+f_2\) is \(O(g)\).
\end{corollary}

\begin{theorem}
  Multiplication of two \(n\)-bit integers can be performed using
  \(O(n^{\log_23})\) bit operations.
\end{theorem}

\begin{theorem}
  Given a positive number \(\epsilon>0\), there is an algorithm for
  multiplication of two \(n\)-bit integers using \(O(n^{1+\epsilon})\) bit
  operations.
\end{theorem}

\begin{theorem}
  There is an algorithm to multiply two \(n\)-bit integers using
  \(O(n\log_2n\log_2\log_2n)\) bit operations.
\end{theorem}

\begin{theorem}
  There is an algorithm to find the quotient \(q=[a/b]\), when the
  \(2n\)-bit integer \(a\) is divided by the integer \(b\) (having no more
  than \(n\) bits), using \(O(M(n))\) bit operations, where \(M(n)\) is the
  number of bit operations needed to multiply two \(n\)-bit integers.
\end{theorem}

\section{Primes and Greatest Common Divisors}

\subsection{Prime Numbers}

\begin{definition}
  A \emph{prime} is an integer greater than \(1\) that is divisible by no
  positive integers other than \(1\) and itself.
\end{definition}

\begin{definition}
  An integer greater than \(1\) that is not prime is called
  \emph{composite}.
\end{definition}

\Emph{The Infinitude of Primes}

\begin{lemma}
  Every integer greater than \(1\) has a prime divisor.
\end{lemma}

\begin{theorem}
  There are infinitely many primes.
\end{theorem}

\Emph{Finding Primes}

\begin{theorem}
  If \(n\) is a composite integer, then \(n\) has a prime factor not
  exceeding \(\sqrt{n}\).
\end{theorem}

\begin{definition}
  The function \(\pi(x)\), where \(x\) is a positive real number, denotes
  the number of prime not exceeding \(x\).
\end{definition}

\begin{theorem}[Dirichlet's Theorem on Primes in Arithmetic Progressions]
  Suppose that \(a\) and \(b\) are relatively prime positive
  integers. Then the arithmetic progression \(an+b\),
  \(n=1,2,3,\ldots\), contains infinitely many primes.
\end{theorem}
\begin{proof}
  See ref.~\cite{Varilly15}.
\end{proof}

\Emph{The Largest Known Primes} Primes of the form \(2^p-1\), where
\(p\) is prime, are called \emph{Mersenne primes}. Currently, the world
record for the largest prime known is \(2^{43,112,609}-1\).

\subsection*{Exercises}

\begin{itemize}
\item[29.] Show that if \(f(x)=a_nx^n+a_{n-1}x^{n-1}+\cdots+a_1x+a_0\), where
  \(n\ge1\) and the coefficients are integers, then there is a positive
  integer \(y\) such that \(f(y)\) is composite. (\emph{Hint}: Assume that
  \(f(x)=p\) is prime, and show that \(p\) divides \(f(x+kp)\) for all
  integers \(k\). Conclude that there in an integer \(y\) such that
  \(f(y)\) is composite from the fact that a polynomial of degree
  \(n\), \(n>1\), takes on each value at most \(n\) times.)
\end{itemize}



%%%%%%%%%%%%%%%%%%%%%% References %%%%%%%%%%%%%%%%%%%%%%%% 

\begin{bibdiv}
  \begin{biblist}
    \bib{Rosen11}{book}{
      author={Rosen, Kenneth H.},
      title={Elementary Number Theory},
      subtitle={\& its applications},
      edition={Sixth Edition},
      year={2011},
      publisher={Addison-Wesley},
      pages={766}
    }
    \bib{Jones75}{article}{
      author={Jones, James P.},
      title={Diophantine Representation of the Fibonacci Numbers},
      year={1975},
      journal={The Fibonacci Quarterly},
      volume={13},
      number={1},
      pages={84--88},
      publisher={The Fibonacci Association}
    }
    \bib{Varilly15}{webpage}{
      author={Anthony V\'arilly},
      title={Dirichlet's Theorem on Arithmetic Progressions},
      url={https://math.rice.edu/~av15/Files/Dirichlet.pdf}
    }
  \end{biblist}
\end{bibdiv}
  
\end{document}
