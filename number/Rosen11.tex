\documentclass[leqno,12pt]{article}
\usepackage[a4paper,margin=0.5in,footskip=0.5in,bottom=0.75in]{geometry}
\usepackage{authblk}
\usepackage{amsrefs}
\usepackage{graphicx,color}

\makeatletter
\DeclareSizeFunction{sub}{\sub@sfcnt\@font@info}
\makeatother
\usepackage{stix2}
\usepackage{amsmath}
\usepackage{mathtools}
\usepackage{bm}
\usepackage{braket}

\newcommand{\Emph}[1]{\textsf{\Large #1\ \ }}

\usepackage{amsthm}
\renewcommand{\qedsymbol}{\tiny$\blacksquare$}
\newcommand{\noqed}{\renewcommand{\qed}{}}
\newcommand*{\newqedtheoremstar}[1]{%
  \AddToHook{env/#1/begin}{\pushQED{\qed}}%
  \AddToHook{env/#1/end}{\popQED}%
  \newtheorem*{#1}}
\newcommand*{\newqedtheorem}[1]{%
  \AddToHook{env/#1/begin}{\pushQED{\qed}}%
  \AddToHook{env/#1/end}{\popQED}%
  \newtheorem{#1}}
\newtheoremstyle{plain}% name
  {6pt}%        Space above
  {6pt}%        Space below
  {\normalfont}% Body font
  {}%           Indent amount
  {\bfseries}%  Head font
  {.}%           Punctuation after head
  {1.0em}%        Space after head
  {\thmname{#1}\thmnumber{ #2}\thmnote{. \ \itshape #3}}% Head spec
\newqedtheoremstar{definition}{Definition}
\newqedtheorem{theorem}{Theorem}[section]

\usepackage{titlesec}
\titleformat{\section}{\sffamily\Huge}{\thesection}{0.5em}{}
\titleformat{\subsection}{\sffamily\bfseries\Large}{\thesubsection}{0.5em}{}
\titleformat{\subsubsection}{\sffamily\bfseries\large}{}{0pt}{}
\usepackage[scheme=plain,UTF8]{ctex}

\usepackage{hyperref}
\begin{document}
\title{初等数论笔记}
\author{邓卫真}
\affil{北京大学物理学院}
\date{2025年11月6日开始}
\maketitle

\section{The Integers}

\subsection{Numbers and Sequences}

\subsubsection{Numbers}

\Emph{The Well-Ordering Property} Every nonempty set of positive
integers has a least element.

\begin{definition}
  The real number \(r\) is \emph{rational} if there are integers \(p\) and
  \(q\), with \(q\ne0\), such that \(r=p/q\). If \(r\) is not rational, it is
  said to be \emph{irrational}.
\end{definition}

\begin{theorem}
  \(\sqrt2\) is irrational.
\end{theorem}

\begin{definition}
  A number \(\alpha\) is \emph{algebraic} if it is the root of a polynomial with
  integer coefficients; that is, \(\alpha\) is algebraic if there exist
  integers \(a_0, a_1, \ldots, a_n\) such that
  \(a_n\alpha^n+a_{n-1}\alpha^{n-1}+\cdots+a_0=0.\) The number
  \(\alpha\) is called \emph{transcendental} if it is not algebraic.
\end{definition}

\subsubsection{The Greatest Integer Function}

\begin{definition}
  The \emph{greatest integer} in a real number \(x\), denoted by
  \([x]\), is the largest integer less than or equal to \(x\). That is,
  \([x]\) is the integer satisfying
  \[
    [x]\le x <[x]+1. \qedhere
  \]
\end{definition}

\begin{definition}
  The \emph{fractional part} of a real number \(x\), denoted by
  \(\{x\}\), is the difference between \(x\) and the greatest integer in
  \(x\), namely, \([x]\). That is, \(\{x\}=x-[x]\).
\end{definition}

\subsubsection{Diophantine Approximation}

\begin{theorem}[The Pigeonhole Principle]
  If \(k+1\) or more objects are placed into \(k\) boxes, then at least
  one box contains two or more of the objects.
\end{theorem}


%%%%%%%%%%%%%%%%%%%%%% References %%%%%%%%%%%%%%%%%%%%%%%% 

\begin{bibdiv}
  \begin{biblist}
    \bib{Rosen11}{book}{
      author={Rosen, Kenneth H.},
      title={Elementary Number Theory},
      subtitle={\& its applications},
      edition={Sixth Edition},
      year={2011},
      publisher={Addison-Wesley},
      pages={766}
    }
  \end{biblist}
\end{bibdiv}
\end{document}
  
\end{document}
